\subsection{Vista}
Le classi della vista sono state implementate utilizzando la libreria
\textbf{Swing} fornita di default dal linguaggio.
\begin{figure}[!ht]
	\centering
	\includegraphics[scale=0.45,center]{core/imgs/UML/VistaBellaUML-noattr.png}
	\caption{Diagramma delle Classi - Vista}
	\label{figure:class_diagram_view}
\end{figure}
Come mostrato dal \textit{class diagram} in Figura \ref{figure:class_diagram_view} 
i 4 frame del gioco estendono un'unica
classe astratta appositamente creata per fornire una veste grafica di default:
l'\textit{AFrame}. Questo è essenzialmente un JFrame con un layout grafico
prestabilito che fornisce metodi utili per l'aggiunta e la sostituzione dei
pannelli che sono di volta in volta visualizzati. È qui che viene impostato il
nome del programma come titolo della finestra, l'icona del gioco e la posizione sul display (al centro).\newline
Così come per i frame, tutti i pannelli estendono da un'unica classe,
l'\textit{APanel}. APanel fornisce anch'esso metodi per l'aggiunta dei vari oggetti che
lo popoleranno (label, aree di testo, bottoni, etc). Ogni frame mantiene al
proprio interno le istanze dei pannelli da mostrare.\newline
Un pattern molto importante e particolarmente sfruttato nell'implementazione
grafica di questo progetto è \textit{Observer}, il quale si basa su uno o più
oggetti (osservatori o observer) che vengono notificati
al verificarsi di determinate condizioni. Ad ogni pannello in grado di
generare eventi è associata un'interfaccia che fornisce i metodi implementati dall'osservatore per ricevere le notifiche.
Dunque, ognuno dei 4 frame si comporta da osservatore nei confronti dei pannelli usati, fornendo
le relative interfacce. Allo stesso modo i frame comunicano con le classi
sottostanti. Il dialogo con la vista avviene dunque in due modi:
\begin{itemize}
	\item quando dal controller (dalle classi Battlesheep e Lobby) si vuole comunicare con i frame, se ne usano i
	metodi che questi forniscono;
	\item quando invece la comunicazione parte dalla grafica, i frame usano i
	metodi forniti dalle rispettive interfacce che gli osservatori avranno il
	compito di implementare.
\end{itemize}



\subsubsection{Main Frame}
\begin{figure}[!h]
	\centering
	\includegraphics[scale=0.4]{core/imgs/gui/main_frame}
	\caption{Main Frame}
	\label{figure:main_frame}
\end{figure}
Il frame principale si compone di un unico pannello, il \textit{MainPanel} (Figura \ref{figure:main_frame}).
Così come discusso in fase di progettazione, tale frame ha il compito di
permettere all'utente la scelta della modalità nella quale avviare il programma:
se come server (lobby) o come nodo (game).\newline
Per la comunicazione verso l'esterno, il MainPanel ha un'interfaccia associata
che utilizza per notificare il MainFrame alla pressione dei bottoni ``Exit'' e
``Start'' il quale, a sua volta, tramite l'observer \textit{MainFrameObserver},
notifca il \textit{Main}.



\subsubsection{Lobby Frame}
\begin{figure}[!h]
	\begin{subfigure}{0.5\textwidth}
		\centering
		\includegraphics[scale=0.4]{core/imgs/gui/lobby_frame}
		\caption{Lobby Frame}
		\label{figure:lobby_frame}
	\end{subfigure}
	\begin{subfigure}{.5\textwidth}
		\centering
		\includegraphics[scale=0.3]{core/imgs/gui/registration_frame}
		\caption{Registration Frame}
		\label{figure:registration_frame}
	\end{subfigure}
	\caption{}
\end{figure}
Il \textit{LobbyFrame} si compone di 3 pannelli, brevemente descritti in seguito. Il primo
pannello permette agli utenti di scegliere il nome della stanza. Ha due bottoni,
uno per l'uscita dal programma ed uno per l'avanzamento al pannello successivo.
Il secondo mostra una gif per l'attesa di connessioni da parte dei peer ed ha
anch'esso un bottone di uscita. Il terzo ed ultimo pannello, visibile in Figura
\ref{figure:lobby_frame}, viene mostrato automaticamente non appena un nodo si
connette. Ospita due pulsanti: uno per l'uscita dal programma ed uno per l'avvio
della stanza.\newline
Quando uno dei bottoni di uscita viene premuto, il frame viene notificato dai pannelli.
Questo, a sua volta, propaga
l'informazione verso il proprio osservatore (\textit{Lobby}) tramite i metodi
forniti dall'interfaccia \textit{LobbyFrameObserver}. Lo stesso comportamento
viene attuato alla pressione del pulsante ``Start''. L'azione sul bottone
``Next'', invece, non viene propagata all'esterno ma serve al frame per sapere
quando visualizzare il secondo pannello.



\subsubsection{Registration Frame}
Il \textit{RegistrationFrame} si compone invece di 7 pannelli.\newline
Il comportamento di questo frame è del tutto simile a quello del LobbyFrame: gli
eventi scatenati dai pulsanti per l'avanzamento della registrazione (``Next'' e
``Previous'') sono usati internamente per la sostituzione dei vari pannelli,
mentre le azioni sui bottoni ``Exit'' e ``Registration'' sono propagate dal
frame verso il proprio osservatore (\textit{Battlesheep}) tramite l'interfaccia
\textit{RegistrationFrameObserver}.\newline
In Figura~\ref{figure:registration_frame} è mostrato il sesto pannello che
permette all'utente di scegliere la posizione delle proprie pecore nel campo di
gioco e di registrarsi alla lobby.



\subsubsection{Game Frame}
\begin{figure}[!h]
	\centering
	\includegraphics[scale=0.4]{core/imgs/gui/game_frame}
	\caption{Game Frame}
	\label{figure:game_frame}
\end{figure}
Il \textit{GameFrame}, mostrato in Figura~\ref{figure:game_frame}, è stato anch'esso
 implementato secondo le specifiche progettuali descritte in precedenza.\newline
La classe \textit{Battlesheep} comunica con questo frame tramite i metodi che
esso mette a disposizione, principalmente per informarlo di come è terminata la
partita per l'utente, per gli altri giocatori e su come si sono conclusi i vari
attacchi. Al contrario, il frame comunica con Battlesheep usando l'interfaccia
\textit{GameFrameObserver} per comunicare la volontà, da parte dell'utente, di
abbandonare il gioco o di attaccare un avversario in una determinata posizione.
