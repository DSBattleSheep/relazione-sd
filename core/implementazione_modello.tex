\subsection{Modello}
Data la natura relativamente semplificata delle regole del gioco, la struttura 
del modello è lineare e ridotta. Per quanto riguarda l'implementazione del 
gioco vera e propria le principali classi coinvolte sono infatti le seguenti:
\begin{itemize}
 \item AField
 \item APlayer
 \item Move
\end{itemize}
Le prime due classi sono astratte, in quanto vengono differenziati il campo di 
gioco proprio al client locale (chiamato MyField) e la propria istanza di 
player(Me). Per quanto riguarda AField, esso rappresenta il campo di gioco e 
contiene pertanto una matrice di interi che memorizza le informazioni 
relative allo stato delle singole celle del campo di battaglia.
La classe astratta APlayer ha due implementazioni: Opponent e Me. Gli 
oggetti di tipo Opponent contengono le informazioni relative agli avversari: ip 
e porta, username e stato dei propri campi di battaglia. La classe Me 
identifica il giocatore locale, memorizza pertanto solo il nome utente e il 
campo del giocatore locale. Le due classi implementano poi alcuni metodi utili 
a conoscere lo stato del giocatore, per sapere ad esempio se egli abbia perso.
L'ultima classe relativa al modello di gioco è Move, che contiene le 
informazioni relative alla mossa effettuata, compreso il risultato di 
quest'ultima. Tale classe è serializzabile, in quanto viene scambiata fra i 
client per comunicare informazioni relative alla mossa scelta e all'ultima 
mossa osservata. Al fine di consentire il recupero di eventuali mosse che non 
siano state trasmesse a tutti i client, ciascuna mossa contiene un intero che 
viene incrementato per ciascuna nuova mossa effettuata.
\\
Oltre alle classi descritte in precedenza, vi è anche una classe NetPlayer, che 
viene utilizzata durante la fase di registrazione per ottenere informazioni 
riguardanti gli altri client dal Lobby Server.