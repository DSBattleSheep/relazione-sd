\subsection{Vista}


Dal punto di vista dell'interfaccia esposta all’utente, il gioco si biforca a seconda che
si avvii il lobby server o il nodo. Nel caso del server, all’amministratore della stanza
viene fornito un pannello che consente l’avvio della partita a sua discrezione. Vengono
inoltre visualizzati username, IP e porta di tutti i peer.
Nel caso del gioco vero e proprio, viene dapprima richiesto all’utente a quale stanza
desideri unirsi (la scelta viene semplificata mostrando una lista di server correntemente attivi).
Proseguendo è stata inserita una animazione che, ironicamente, fa intuire il motivo del nome del gioco.
Successivamente egli immette lo username con cui vorrà registrarsi e la posizione delle pecore sul
proprio campo di battaglia. Il nodo mostra dunque una schermata di attesa, prima
che la lobby gli comunichi i partecipanti e la partita possa iniziare.
Vi è infine la schermata di gioco, che consente all’utente di specificare la posizione e
il giocatore da attaccare. Essa fornisce inoltre uno storico delle mosse effettuate da
tutti gli utenti. È inoltre presente una visualizzazione in forma di griglia del campo
di gioco di tutti i giocatori e del proprio, completo di informazioni su quali celle siano
state colpite e quali contenessero pecore. Un ulteriore aspetto visuale è la gestione
di informazioni riguardanti i giocatori che hanno perso e quelli che hanno subito un
crash e sono pertanto usciti dalla partita. Sono infine visualizzate le informazioni sul
piazzamento quando si perde o finisce la partita.