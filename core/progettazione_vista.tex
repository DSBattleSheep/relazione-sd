\subsection{Vista}



\subsubsection{Main Frame}
\label{subsubsection:progettazione_main_frame}
È il frame principale che parte all'avvio del gioco e chiede in che modalità
avviare il programma, se come server (lobby) o come client (game).\newline
Si compone semplicemente di due radio button che permettono di effettuare la
scelta, confermata tramite la pressione del bottone ``Ok''. La modalità di
default è ``client''.



\subsubsection{Lobby Frame}
\label{subsubsection:progettazione_lobby_frame}
Il frame della lobby permette di raccogliere le iscrizioni al gioco da parte dei
vari client, fornendo loro un unico punto di accesso al programma. È la vista
che viene avviata se si decide di avviare il programma come server e si compone
di 3 pannelli.\newline
Il primo pannello permette di scegliere il nome della stanza. Uuna volta
confermato il nome sarà mostrato il secondo pannello il quale mostra in alto
l'indirizzo ip e la porta sulla quale si è in ascolto e, al centro, un messaggio
che informa l'utente dell'attesa della connessione di eventuali client. Non
appena il primo client si è connesso, viene mostrato il terzo ed ultimo pannello
che ospita, al centro, una tabella che elenca tutti gli host (client) che via
via si connettono alla lobby. A discrezione dell'utente, tramite la pressione
del bottone ``Start'', saràpossibile chiudere le registrazioni ed avviare il
gioco.



\subsubsection{Registration Frame}
\label{subsubsection:progettazione_registration_frame}
Il frame per la registrazione è la vista che viene mostrata se si decide di
avviare il programma come client e si compone di una serie di pannelli che
possono essere raggruppati in base alle funzionalità che offrono.\newline
Nel primo pannello sono elencati i nomi delle stanze ed i relativi indirizzi ip
degli host che eseguono il server (lobby). Questo pannello permette all'utente
di scegliere a quale partita iscriversi. È inoltre fornita, tramite la pressione
di un bottone, la possibilità di specificare manualmente un indirizzo ip. Una
volta confermata la propria scelta, saranno mostrati una serie di pannelli che
emulano il loading del gioco, un warning panel che mostra un messaggio di alert
che avvisa l'utente del fatto che il gioco sta venendo craccato ed un ultimo
pannello che mostra una pecora nell'immagine di background, dove la scritta
``Battleship'' è stata rimpiazzata con la scritta ``Battlesheep''. Dopo la
sequenza di pannelli che introducono l'utente al gioco, gli viene richiesto
l'username. La terza ed ultima informazione richiesta è la posizione delle
pecore. A questo fine è mostrato un pannello con, al centro, la rappresentazione
del campo di gioco. In alto è presente un numero che indica quante pecore
bisogna ancora inserire. Una volta confermate tutte le informazioni necessarie,
tramite la pressione del bottone ``Registration'' sarà mostrato un ultimo
pannello che informa l'utente dell'avvenuta registrazione ed attesa dell'avvio
del gioco.\newline
Come è possibile intuire, il frame per la registrazione serve a raccogliere
tutte le informazioni necessarie al gioco e a comunicare il proprio indirizzo
ip ed username del giocatore alla lobby (entry point).



\subsubsection{Game Frame}
\label{subsubsection:progettazione_game_frame}
Questo frame viene avviato una volta completata la registrazione e permette agli
utenti di giocare gli uni contro gli altri.\newline
Nella parte alta è presente un'immagine con il titolo del gioco, quella centrale
è occupata a sinistra dal campo di gioco dell'utente e a destra da quello
dell'avversario selezionato, a destra è presente una lista che elenca i nomi di
tutti i giocatori mentre, nella sezione bassa del frame, è presente un'area di
testo che funziona come una sorta di log.
