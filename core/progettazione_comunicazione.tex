\subsection{Comunicazione}
La comunicazione fra i client e fra client e server avviene in tre fasi:
\begin{itemize}
 \item Registrazione;
 \item Scelta del turno;
\item Gioco.
\end{itemize}
Ciascun client ha la necessità di localizzare l'indirizzo della lobby a cui 
registrarsi. Per effettuare tale ricerca si è optato per un approccio forse non 
molto elegante ma di certa efficacia: i client scansionano la sottorete (con 
maschera 255.255.255.0) alla ricerca di lobby server attivi, che rispondono con 
il proprio nome identificativo.
Durante la fase di registrazione i client si connettono al 
server fornendo il proprio username e la porta su cui sono in ascolto. Il 
server risponde con nomi utenti, porte e IP degli altri client in modo da 
consentire l'avvio della comunicazione di tipo p2p. In questa fase non vengono 
gestiti eventuali crash dei client, in quanto tale gestione avverrà nelle fasi 
successive, nelle quali viene implementata la comunicazione distribuita vera e 
propria.
\\
Durante la scelta del turno ciascun client scorre la lista dei propri 
avversari e contatta ognuno di essi per ottenere un numero generato in modo 
pseudocausale. In questo modo i client sono in possesso di valori associati a 
ciascun avversario, che vengono utilizzati per determinare l'ordine di gioco in 
modo fair. Nel caso in cui due o più valori siano uguali, l'ordine fra di essi 
viene determinato a partire dall'ordinamento alfabetico degli username. In 
seguito alla scelta del turno ciascun client è in possesso di una sequenza 
ordinata di giocatori, che viene usata in seguito per scegliere chi debba 
giocare.
\\
Può quindi iniziare la fase di gioco vero e proprio. Tutti i client si 
collegano quindi al giocatore che possiede il turno (turn owner). %FIXME
Quest'ultimo seleziona dunque la propria mossa, la comunica al peer bersaglio, 
allo scopo di conoscere lo stato della cella selezionata, infine comunica la 
mossa ed il risultato a tutti gli avversari. Per gestire eventuali crash 
da parte di utenti diversi dal turn owner, viene restituita ai client anche una 
lista di player crashati, che viene utilizzata in seguito per modificare in 
modo appropriato le strutture dati relative ai giocatori. Oltre al caso di crash 
veri e propri, viene anche gestito il caso in cui la comunicazione di rete a 
partire da qualche giocatore sia troppo latente, per cui viene impostato un 
timer oltre il quale non vengono accettate ulteriori connessioni.
Nel caso in cui sia invece il turn owner a subire un crash, gli altri giocatori 
vengono allertati immediatamente e il gioco può procedere con il client 
successivo.
Un'ultima possibilità di errore emerge nel caso in cui il turn owner crashi 
mentre sta restituendo la mossa effettuata ai client. In tal caso, ci 
si potrebbe trovare in una situazione di inconsistenza dello stato globale del 
sistema. Alcuni client conoscerebbero la mossa che il precedente turn owner ha 
effettuato, altri no. È pertanto necessario gestire tali situazioni per 
ripristinare la coerenza globale: al turno successivo sarà il nuovo turn owner 
a confrontare gli orologi logici contenuti in ciascuna mossa al fine di 
stabilire quale sia la più recente. Dopo aver aggiornato il proprio stato se 
necessario, il turn owner comunica agli avversari non aggiornati la mossa più 
recente, prima di iniziare il turno vero e proprio. In questo modo è garantita 
la coerenza dello stato condiviso prima che possa avvenire qualsiasi tipo di 
mossa.
