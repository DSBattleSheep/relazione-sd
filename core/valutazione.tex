I test da noi effettuati hanno evidenziato la capacità del software di gestire 
guasti di tipo crash su uno qualsiasi dei nodi durante la comunicazione. Se 
la gestione di return parziali ha causato un leggero aumento della complessità 
del codice, tali errori non causano più inconsistenze durature, permettendo ai 
peer di accordarsi autonomamente rispetto all'ordine delle mosse effettuate.
\\
L'implementazione della discovery delle lobby da parte dei nodi consente 
inoltre di fornire all'utente una semplice scelta delle stanze di gioco, senza 
assumere che essi conoscano a priori l'indirizzo IP. Naturalmente 
tale funzionalità è utile solo nel caso in cui i peer si connettano ad un IP 
della propria sottorete, in quanto è impensabile fare discovery fra reti 
eterogenee senza prevedere un qualche server di appoggio esterno.
\\
Nonostante la semplicità del gioco scelto per il progetto, il protocollo di 
comunicazione implementato può essere facilmente generalizzabile a qualsiasi 
gioco a turni che proceda con una mossa per giocatore in un ordine prestabilito.
Sarebbe infatti sufficiente modificare le strutture dati relative alle mosse 
effettuate e cambiare leggermente le altre parti del codice per implementare un 
qualsiasi nuovo gioco a turni.
\\
Per aumentare la godibilità del gioco si potrebbe infine imporre al giocatore 
di inserire le pecore sul campo di gioco in celle contigue, in modo analogo a 
quanto avviene nella battaglia navale tradizionale. In questo modo ciascun 
attacco che ha avuto successo causerà una ricerca delle pecore rimanenti nelle celle adiacenti,
migliorando la gradevolezza complessiva del gioco.