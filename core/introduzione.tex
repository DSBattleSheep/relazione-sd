Nell'ambito dei giochi multiplayer online è interessante osservare come sia 
possibile sfruttare paradigmi di comunicazione di tipo \textit{p2p} che non
prevedono, quindi, un server centrale che coordini la comunicazione fra client.
Tale approccio introduce certamente maggiori difficoltà implementative, ma
non esponendo un singolo point-of-failure risulta essere più resistente ad
eventuali problemi di rete o sugli host.\newline
Il progetto qui descritto riguarda un'estensione multiplayer del famoso gioco
della \textit{Battaglia Navale} tradizionale, che sfrutti dunque il paradigma di
comunicazione p2p.\newline
La battaglia navale tradizionale prevede due giocatori che dapprima piazzano 
delle navi su una griglia bidimensionale di dimensioni prestabilite. Durante la 
fase di gioco vera e propria i due partecipanti si alternano dichiarando 
attacchi sulla griglia dell'avversario.\newline
Nella versione da noi implementata vi sono alcune modifiche rispetto alla 
versione tradizionale. Le navi sono state rimpiazzate da pecore sfruttando il 
gioco di parole fra battleship (battaglia navale) e battlesheep (battaglia fra
pecore). Al fine di semplificare le regole del gioco, poi, ciascuna pecora viene
piazzata singolarmente sulla griglia occupando, dunque, una sola cella a
differenza delle navi che avrebbero occupato più spazio.\newline
L'estensione più importante riguarda però l'aspetto multigiocatore: ciascun 
partecipante dichiara, durante il suo turno, l'avversario da colpire e la 
cella obiettivo del suo attacco. Il risultato viene così comunicato a tutti i 
giocatori. Il vincitore del gioco è l'ultimo partecipante ad avere pecore vive 
sul proprio campo di battaglia. La classifica finale è data dall'ordine nel 
quale i giocatori perdono tutte le proprie pecore.
