Nell'ambito dei giochi multiplayer online è interessante osservare come sia 
possibile sfruttare paradigmi di comunicazione di tipo P2P, che non prevedano 
quindi un singolo server centrale che coordini la comunicazione fra client. 
Tale approccio introduce certamente maggiori difficoltà implementative, ma 
non esponendo un singolo point-of-failure, è maggiormente resistente ad 
eventuali problemi di rete o sugli host.
\\
Il progetto qui descritto riguarda un'estensione multiplayer della battaglia 
navale tradizionale, che sfrutti il paradigma di comunicazione p2p.
\subsection{Il gioco}
La battaglia navale tradizionale prevede due giocatori che dapprima piazzano 
delle navi su una griglia bidimensionale di dimensioni prestabilite. Durante la 
fase di gioco vera e propria i due partecipanti si alternano dichiarando 
attacchi sulla griglia dell'avversario. Quest'ultimo risponde:
  \begin{itemize}
   \item \emph{acqua} se l'attacco è andato a vuoto;
   \item \emph{colpito} se l'attacco ha colpito una nave dell'avversario, senza 
però affondarla;
   \item \emph{colpito e affondato} se l'attacco ha colpito una nave 
dell'avversario, affondandola.
  \end{itemize}
Nella versione da noi implementata vi sono alcune modifiche rispetto alla 
versione tradizionale. Le navi sono state rimpiazzate da pecore, sfruttando il 
gioco di parole fra ship e sheep: battleship (battaglia navale) è diventato 
battlesheep (battaglia fra pecore).
Al fine di semplificare le regole del gioco, ciascuna pecora viene piazzata 
individualmente sulla griglia occupando dunque una sola cella, a differenza 
delle navi che avrebbero occupato più spazio.
\\
L'estensione più importante riguarda l'aspetto multigiocatore: ciascun 
partecipante dichiara, durante il suo turno, l'avversario da colpire e la 
cella obiettivo del suo attacco. Il risultato viene poi comunicato a tutti i 
giocatori. Il vincitore del gioco è l'ultimo partecipante ad avere pecore vive 
sul proprio campo di battaglia. La classifica finale è data dall'ordine nel 
quale i giocatori perdono tutte le proprie pecore.