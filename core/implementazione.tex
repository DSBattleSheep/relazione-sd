Per l'implementazione del progetto si è scelto di utilizzare il linguaggio 
di programmazione basato sugli oggetti Java, che tramite il framework RMI 
fornisce API specificamente progettate per consentire la realizzazione di 
applicazioni distribuite in modo efficace. Per quanto riguarda l'interfaccia 
grafica, si è utilizzata la libreria di Java Swing.
\\
Il software è stato implementato secondo il pattern architetturale \textbf{MVC},
anche noto come \textbf{Model-View-Controller}, il quale permette di creare una
architettura cosiddetta \textit{multi-tier}, ovvero nella quale le tre
componenti fondamentali di un programma sono suddivise in base ai compiti che
svolgono:
\begin{itemize}
	\item il \textit{modello}: che fornisce i metodi per accedere ai dati utili
	dell'applicazione;
	\item la \textit{vista}: la quale visualizza i dati del modello e si 
occupa
	dell'interazione con l'utente;
	\item il \textit{controller}: che riceve i comandi dall'utente e li attua
	modificando lo stato degli altri due componenti.
\end{itemize}
Nel nostro specifico caso, il modello si compone di tutte quelle classi atte a
mantenere i dati dell'applicazione. Abbiamo dunque l'oggetto che rappresenta il
campo di gioco dell'utente e degli avversari, l'oggetto che rappresenta i
giocatori impegnati nella partita, quello che astrae i dati degli host che si
registrano alla lobby con l'intenzione di partecipare al gioco,
eccetera\dots\newline
La vista è invece formata da tutta una serie di classi atte a permettere la
corretta visualizzazione dei dati, a partire dal frame per la registrazione fino
ad arrivare al frame di gioco.\newline
Vista la natura distribuita del programma, in questo specifico caso il
controller può essere pensato come a tutti quei meccanismi di interazione che
partono dall'utente ed hanno come obiettivo quello di modificare lo
\textit{stato globale} del gioco. Dunque, non soltanto le componenti software
che permettono di interagire direttamente con i dati mantenuti negli oggetti
del mio modello (locale), ma anche tutte le classi che permettono la
comunicazione fra gli host impegnati nella partita.\newline