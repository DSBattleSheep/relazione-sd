Le principali componenti del software in esame sono due: un server di 
registrazione (lobby) e i peer con i quali interagiscono i giocatori.
Il server di registrazione ha lo scopo di fornire un unico punto di accesso, 
in maniera da consentire ai nodi di conoscere le informazioni rilevanti degli 
altri peer. Una volta completato il suo compito, il server termina e tutte le 
comunicazioni successive avvengono con paradigma p2p fra i singoli nodi.
La conoscenza dei singoli peer è fortemente limitata: essi conoscono infatti 
solo la posizione delle pecore sul proprio campo di gioco, non quelle altrui. 
Vengono invece comunicate a tutti le mosse effettuate da ciascun giocatore e 
quale effetto esse abbiano avuto sul campo degli avversari.
\\
Rispetto all'interazione con l'utente, i peer forniscono la possibilità di 
specificare l'indirizzo IP su cui risiede il server della lobby, uno username 
che identifichi univocamente il giocatore e la posizione delle proprie pecore 
nel campo di gioco.