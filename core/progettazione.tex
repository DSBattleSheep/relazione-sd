Le principali componenti del software in esame sono due: un server di 
registrazione (lobby) e i client con i quali interagiscono i giocatori.
Il server di registrazione ha lo scopo di fornire un unico punto di accesso, 
in maniera da consentire ai client di conoscere le informazioni rilevanti degli 
altri peer. Una volta completato il suo compito, il server termina e tutte le 
comunicazioni successive avvengono con paradigma p2p fra i singoli client.
La conoscenza dei singoli peer è fortemente limitata: essi conoscono infatti 
solo la posizione delle pecore sul proprio campo di gioco, non quelle altrui. 
Vengono invece comunicate a tutti le mosse effettuate da ciascun giocatore e 
quale effetto esse abbiano avuto sul campo degli avversari.
\\
Rispetto all'interazione con l'utente, i client forniscono la possibilità di 
specificare l'indirizzo IP su cui risiede il server della lobby, uno username 
che identifichi univocamente il giocatore e la posizione delle proprie pecore 
nel campo di gioco.
\\
Passiamo ora ad una trattazione più approfondita del protocollo di 
comunicazione da noi progettato.



\subsection{La comunicazione}
La comunicazione fra i client e fra client e server avviene in tre fasi:
\begin{itemize}
 \item Registrazione;
 \item Scelta del turno;
\item Gioco.
\end{itemize}
Ciascun client ha la necessità di localizzare l'indirizzo della lobby a cui 
registrarsi. Per effettuare tale ricerca si è optato per un approccio forse non 
molto elegante ma di certa efficacia: i client scansionano la sottorete (con 
maschera 255.255.255.0) alla ricerca di lobby server attivi, che rispondono con 
il proprio nome identificativo.
Durante la fase di registrazione i client si connettono al 
server fornendo il proprio username e la porta su cui sono in ascolto. Il 
server risponde con nomi utenti, porte e IP degli altri client in modo da 
consentire l'avvio della comunicazione di tipo p2p. In questa fase non vengono 
gestiti eventuali crash dei client, in quanto tale gestione avverrà nelle fasi 
successive, nelle quali viene implementata la comunicazione distribuita vera e 
propria.
\\
Durante la scelta del turno ciascun client scorre la lista dei propri 
avversari e contatta ognuno di essi per ottenere un numero generato in modo 
pseudocausale. In questo modo i client sono in possesso di valori associati a 
ciascun avversario, che vengono utilizzati per determinare l'ordine di gioco in 
modo fair. Nel caso in cui due o più valori siano uguali, l'ordine fra di essi 
viene determinato a partire dall'ordinamento alfabetico degli username. In 
seguito alla scelta del turno ciascun client è in possesso di una sequenza 
ordinata di giocatori, che viene usata in seguito per scegliere chi debba 
giocare.
\\
Può quindi iniziare la fase di gioco vero e proprio. Tutti i client si 
collegano quindi al giocatore che possiede il turno (turn owner). %FIXME
Quest'ultimo seleziona dunque la propria mossa, la comunica al peer bersaglio, 
allo scopo di conoscere lo stato della cella selezionata, infine comunica la 
mossa ed il risultato a tutti gli avversari. Per gestire eventuali crash 
da parte di utenti diversi dal turn owner, viene restituita ai client anche una 
lista di player crashati, che viene utilizzata in seguito per modificare in 
modo appropriato le strutture dati relative ai giocatori. Oltre al caso di crash 
veri e propri, viene anche gestito il caso in cui la comunicazione di rete a 
partire da qualche giocatore sia troppo latente, per cui viene impostato un 
timer oltre il quale non vengono accettate ulteriori connessioni.
Nel caso in cui sia invece il turn owner a subire un crash, gli altri giocatori 
vengono allertati immediatamente e il gioco può procedere con il client 
successivo.
Un'ultima possibilità di errore emerge nel caso in cui il turn owner crashi 
mentre sta restituendo la mossa effettuata ai client. In tal caso, ci 
si potrebbe trovare in una situazione di inconsistenza dello stato globale del 
sistema. Alcuni client conoscerebbero la mossa che il precedente turn owner ha 
effettuato, altri no. È pertanto necessario gestire tali situazioni per 
ripristinare la coerenza globale: al turno successivo sarà il nuovo turn owner 
a confrontare gli orologi logici contenuti in ciascuna mossa al fine di 
stabilire quale sia la più recente. Dopo aver aggiornato il proprio stato se 
necessario, il turn owner comunica agli avversari non aggiornati la mossa più 
recente, prima di iniziare il turno vero e proprio. In questo modo è garantita 
la coerenza dello stato condiviso prima che possa avvenire qualsiasi tipo di 
mossa.



\subsection{GUI}
La grafica del gioco si compone di 4 \textit{frame} principali:
\begin{itemize}
	\item il \textit{main frame}: è il frame principale che parte all'avvio del
	gioco e chiede in che modalità avviare il programma, se come server (lobby)
	o come client (game);
	\item il \textit{lobby frame}: è il frame che viene avviato se si decide di
	eseguire il gioco come server e fornisce ai client un punto di accesso unico
	al programma;
	\item il \textit{registration frame}: è il frame che viene avviato se si
	decide di eseguire il gioco come client e permette all'utente di effettuare
	la registrazione;
	\item il \textit{game frame}: è il frame che viene avviato una volta
	completata la registrazione e permette all'utente di giocare.
\end{itemize}

\subsubsection{Main Frame}
Questo frame si compone semplicemente di due radio button che permettono di
scegliere la modalità nella quale avviare il gioco. La modalità di default è
quella che permette di far partire il programma come client. Una volta
effettuata la scelta sarà sufficiente confermare l'azione premendo sul bottone
``Ok''.

\subsubsection{Lobby Frame}
Il frame della lobby permette di raccogliere le iscrizioni al gioco da parte dei
vari client e si compone di 3 pannelli.\newline
Il primo pannello permette di scegliere il nome della stanza. Uuna volta
confermato il nome sarà mostrato il secondo pannello il quale mostra in alto
l'indirizzo ip e la porta sulla quale si è in ascolto e, al centro, un messaggio
che informa l'utente dell'attesa della connessione di eventuali client. Non
appena il primo client si è connesso, viene mostrato il terzo ed ultimo pannello
che ospita, al centro, una tabella che elenca tutti gli host (client) che via
via si connettono alla lobby. A discrezione dell'utente, tramite la pressione
del bottone ``Start'', saràpossibile chiudere le registrazioni ed avviare il
gioco.

\subsubsection{Registration Frame}
Il frame per la registrazione si compone di una serie di pannelli che possono
essere raggruppati in base alle funzionalità che offrono.\newline
Nel primo pannello sono elencati i nomi delle stanze ed i relativi indirizzi ip
degli host che eseguono il server (lobby). Questo pannello permette all'utente
di scegliere a quale partita iscriversi. È inoltre fornita, tramite la pressione
di un bottone, la possibilità di specificare manualmente un indirizzo ip. Una
volta confermata la propria scelta, saranno mostrati una serie di pannelli che
emulano il loading del gioco, un warning panel che mostra un messaggio di alert
che avvisa l'utente del fatto che il gioco sta venendo craccato ed un ultimo
pannello che mostra una pecora nell'immagine di background, dove la scritta
``Battleship'' è stata rimpiazzata con la scritta ``Battlesheep''. Dopo la
sequenza di pannelli che introducono l'utente al gioco, gli viene richiesto
l'username. La terza ed ultima informazione richiesta è la posizione delle
pecore. A questo fine è mostrato un pannello con, al centro, la rappresentazione
del campo di gioco. In alto è presente un numero che indica quante pecore
bisogna ancora inserire. Una volta confermate tutte le informazioni necessarie,
tramite la pressione del bottone ``Registration'' sarà mostrato un ultimo
pannello che informa l'utente dell'avvenuta registrazione ed attesa dell'avvio
del gioco.\newline
Come è possibile intuire, il frame per la registrazione serve a raccogliere
tutte le informazioni necessarie al gioco e a comunicare il proprio indirizzo
ip ed username del giocatore alla lobby (entry point).

\subsubsection{Game Frame}
Questo frame permette agli utenti di giocare gli uni contro gli altri. Nella
parte alta è presente un'immagine con il titolo del gioco, quella centrale è
occupata a sinistra dal campo di gioco dell'utente e a destra da quello
dell'avversario selezionato, a destra è presente una lista che elenca i nomi di
tutti i giocatori mentre, nella sezione bassa del frame, è presente un'area di
testo che funziona come una sorta di log.
