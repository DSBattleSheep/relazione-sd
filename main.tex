\documentclass[a4paper,10pt]{scrartcl}
\usepackage[utf8]{inputenc}
\usepackage[italian]{babel}

\usepackage{hyperref}
\hypersetup{
	colorlinks=true,
	linkcolor=red,
	citecolor=red,
	filecolor=red,
	urlcolor=red
}

\usepackage{graphicx}
\usepackage{caption}
\usepackage{subcaption}

\usepackage[export]{adjustbox}[2011/08/13]



\begin{document}

% title
\title{BattleSheep}
\subtitle{Progetto di Sistemi Distribuiti}
\author{Giulio Biagini, Michele Corazza, Gianluca Iselli}
\maketitle

% abstract
\begin{abstract}
In questa relazione tratteremo la realizzazione di una versione della battaglia
navale multigiocatore. Per estendere il gioco nella sua versione tradizionale si sono rese
necessarie delle modifiche alle sue regole. 
Tale sistema dovrà risultare distribuito, con particolare attenzione
alla tolleranza ai guasti di tipo crash sui nodi, che dovranno essere gestiti
in modo appropriato per non compromettere la partita.
\end{abstract}

% sections
\section{Introduzione}
Nell'ambito dei giochi multiplayer online è interessante osservare come sia 
possibile sfruttare paradigmi di comunicazione di tipo \textit{p2p} che non
prevedono, quindi, un server centrale che coordini la comunicazione fra client.
Tale approccio introduce certamente maggiori difficoltà implementative, ma
non esponendo un singolo point-of-failure risulta essere più tollerante ad
eventuali problemi di rete o sugli host.\newline
Il progetto qui descritto riguarda un'estensione multiplayer del famoso gioco
della \textit{Battaglia Navale} tradizionale, che sfrutti dunque il paradigma di
comunicazione p2p.\newline
La versione tradizionale del gioco prevede due giocatori che dapprima piazzano 
delle navi su una griglia bidimensionale di dimensioni prestabilite. Durante la 
partita vera e propria i due partecipanti si alternano dichiarando 
attacchi sulla griglia dell'avversario.\newline
Nella versione da noi implementata sono state apportate alcune modifiche.
Le navi sono state rimpiazzate da pecore sfruttando il 
gioco di parole fra battleship (battaglia navale) e battlesheep (battaglia fra
pecore). Al fine di semplificare le regole del gioco, poi, ciascuna pecora viene
piazzata singolarmente sulla griglia occupando, dunque, una sola cella, a
differenza delle navi che avrebbero occupato più spazio.\newline
L'estensione più importante riguarda però l'aspetto multigiocatore: ciascun 
partecipante dichiara, durante il suo turno, l'avversario da colpire e la 
cella obiettivo del suo attacco. Il risultato viene così comunicato a tutti i 
giocatori. Il vincitore del gioco è l'ultimo partecipante ad avere pecore vive 
sul proprio campo di battaglia. La classifica finale è data dall'ordine nel 
quale i giocatori perdono tutte le proprie pecore.


\section{Aspetti Progettuali}
Le principali componenti del software in esame sono due: un server di 
registrazione (lobby) e i client con i quali interagiscono i giocatori.
Il server di registrazione ha lo scopo di fornire un unico punto di accesso, 
in maniera da consentire ai client di conoscere le informazioni rilevanti degli 
altri peer. Una volta completato il suo compito, il server termina e tutte le 
comunicazioni successive avvengono con paradigma p2p fra i singoli client.
La conoscenza dei singoli peer è fortemente limitata: essi conoscono infatti 
solo la posizione delle pecore sul proprio campo di gioco, non quelle altrui. 
Vengono invece comunicate a tutti le mosse effettuate da ciascun giocatore e 
quale effetto esse abbiano avuto sul campo degli avversari.
\\
Rispetto all'interazione con l'utente, i client forniscono la possibilità di 
specificare l'indirizzo IP su cui risiede il server della lobby, uno username 
che identifichi univocamente il giocatore e la posizione delle proprie pecore 
nel campo di gioco.
\\
Passiamo ora ad una trattazione più approfondita del protocollo di 
comunicazione da noi progettato.

\subsection{La comunicazione}
La comunicazione fra i client e fra client e server avviene in tre fasi:
\begin{itemize}
 \item Registrazione;
 \item Scelta del turno;
\item Gioco.
\end{itemize}

Durante la fase di registrazione i client si connettono al 
server fornendo il proprio username e la porta su cui sono in ascolto. Il 
server risponde con nomi utenti, porte e IP degli altri client in modo da 
consentire l'avvio della comunicazione di tipo p2p. In questa fase non vengono 
gestiti eventuali crash dei client, in quanto tale gestione avverrà nelle fasi 
successive, nelle quali viene implementata la comunicazione distribuita vera e 
propria.
\\
Durante la scelta del turno ciascun client scorre la lista dei propri 
avversari e contatta ognuno di essi per ottenere un numero generato in modo 
pseudocausale. In questo modo i client sono in possesso di valori associati a 
ciascun avversario, che vengono utilizzati per determinare l'ordine di gioco in 
modo fair. Nel caso in cui due o più valori siano uguali, l'ordine fra di essi 
viene determinato a partire dall'ordinamento alfabetico degli username. In 
seguito alla scelta del turno ciascun client è in possesso di una sequenza 
ordinata di giocatori, che viene usata in seguito per scegliere chi debba 
giocare.
\\
Può quindi iniziare la fase di gioco vero e proprio. Tutti i client si 
collegano quindi al giocatore che possiede il turno (turn owner). %FIXME
Quest'ultimo seleziona dunque la propria mossa, la comunica al peer bersaglio, 
allo scopo di conoscere lo stato della cella selezionata, infine comunica la 
mossa ed il risultato a tutti gli avversari. Per gestire eventuali crash 
da parte di utenti diversi dal turn owner, viene restituita ai client anche una 
lista di player crashati, che viene utilizzata in seguito per modificare in 
modo appropriato le strutture dati relative ai giocatori. Oltre al caso di crash 
veri e propri, viene anche gestito il caso in cui la comunicazione di rete a 
partire da qualche giocatore sia troppo latente, per cui viene impostato un 
timer oltre il quale non vengono accettate ulteriori connessioni.
Nel caso in cui sia invece il turn owner a subire un crash, gli altri giocatori 
vengono allertati immediatamente e il gioco può procedere con il client 
successivo.
Un'ultima possibilità di errore emerge nel caso in cui il turn owner crashi 
mentre sta restituendo la mossa effettuata ai client. In tal caso, ci 
si potrebbe trovare in una situazione di inconsistenza dello stato globale del 
sistema. Alcuni client conoscerebbero la mossa che il precedente turn owner ha 
effettuato, altri no. È pertanto necessario gestire tali situazioni per 
ripristinare la coerenza globale: al turno successivo sarà il nuovo turn owner 
a confrontare gli orologi logici contenuti in ciascuna mossa al fine di 
stabilire quale sia la più recente. Dopo aver aggiornato il proprio stato se 
necessario, il turn owner comunica agli avversari non aggiornati la mossa più 
recente, prima di iniziare il turno vero e proprio. In questo modo è garantita 
la coerenza dello stato condiviso prima che possa avvenire qualsiasi tipo di 
mossa.
\subsection{GUI}
La grafica del gioco si compone di 4 \textit{frame} principali:
\begin{itemize}
	\item il \textit{main frame}: è il frame principale che parte all'avvio del
	gioco e chiede in che modalità avviare il programma, se come server (lobby)
	o come client (game);
	\item il \textit{lobby frame}: è il frame che viene avviato se si decide di
	eseguire il gioco come server e fornisce ai client un punto di accesso unico
	al programma;
	\item il \textit{registration frame}: è il frame che viene avviato se si
	decide di eseguire il gioco come client e permette all'utente di effettuare
	la registrazione;
	\item il \textit{game frame}: è il frame che viene avviato una volta
	completata la registrazione e permette all'utente di giocare.
\end{itemize}

\subsubsection{Main Frame}
\label{subsubsection:progettazione_main_frame}
Questo frame si compone semplicemente di due radio button che permettono di
scegliere la modalità nella quale avviare il gioco. La modalità di default è
quella che permette di far partire il programma come client. Una volta
effettuata la scelta sarà sufficiente confermare l'azione premendo sul bottone
``Ok''.

\subsubsection{Lobby Frame}
\label{subsubsection:progettazione_lobby_frame}
Il frame della lobby permette di raccogliere le iscrizioni al gioco da parte dei
vari client e si compone di 3 pannelli.\newline
Il primo pannello permette di scegliere il nome della stanza. Uuna volta
confermato il nome sarà mostrato il secondo pannello il quale mostra in alto
l'indirizzo ip e la porta sulla quale si è in ascolto e, al centro, un messaggio
che informa l'utente dell'attesa della connessione di eventuali client. Non
appena il primo client si è connesso, viene mostrato il terzo ed ultimo pannello
che ospita, al centro, una tabella che elenca tutti gli host (client) che via
via si connettono alla lobby. A discrezione dell'utente, tramite la pressione
del bottone ``Start'', saràpossibile chiudere le registrazioni ed avviare il
gioco.

\subsubsection{Registration Frame}
\label{subsubsection:progettazione_registration_frame}
Il frame per la registrazione si compone di una serie di pannelli che possono
essere raggruppati in base alle funzionalità che offrono.\newline
Nel primo pannello sono elencati i nomi delle stanze ed i relativi indirizzi ip
degli host che eseguono il server (lobby). Questo pannello permette all'utente
di scegliere a quale partita iscriversi. È inoltre fornita, tramite la pressione
di un bottone, la possibilità di specificare manualmente un indirizzo ip. Una
volta confermata la propria scelta, saranno mostrati una serie di pannelli che
emulano il loading del gioco, un warning panel che mostra un messaggio di alert
che avvisa l'utente del fatto che il gioco sta venendo craccato ed un ultimo
pannello che mostra una pecora nell'immagine di background, dove la scritta
``Battleship'' è stata rimpiazzata con la scritta ``Battlesheep''. Dopo la
sequenza di pannelli che introducono l'utente al gioco, gli viene richiesto
l'username. La terza ed ultima informazione richiesta è la posizione delle
pecore. A questo fine è mostrato un pannello con, al centro, la rappresentazione
del campo di gioco. In alto è presente un numero che indica quante pecore
bisogna ancora inserire. Una volta confermate tutte le informazioni necessarie,
tramite la pressione del bottone ``Registration'' sarà mostrato un ultimo
pannello che informa l'utente dell'avvenuta registrazione ed attesa dell'avvio
del gioco.\newline
Come è possibile intuire, il frame per la registrazione serve a raccogliere
tutte le informazioni necessarie al gioco e a comunicare il proprio indirizzo
ip ed username del giocatore alla lobby (entry point).

\subsubsection{Game Frame}
\label{subsubsection:progettazione_game_frame}
Questo frame permette agli utenti di giocare gli uni contro gli altri. Nella
parte alta è presente un'immagine con il titolo del gioco, quella centrale è
occupata a sinistra dal campo di gioco dell'utente e a destra da quello
dell'avversario selezionato, a destra è presente una lista che elenca i nomi di
tutti i giocatori mentre, nella sezione bassa del frame, è presente un'area di
testo che funziona come una sorta di log.

\subsection{Comunicazione}
La comunicazione fra i nodi e fra essi e i server di registrazione avviene nelle tre fasi seguenti:

\subsubsection{Registrazione}
Ciascun peer ha la necessità di localizzare l'indirizzo della lobby a cui 
registrarsi. Per effettuare tale ricerca si è optato per un approccio forse non 
molto elegante ma di sicura efficacia: i nodi scansionano la sottorete (con 
maschera 255.255.255.0) alla ricerca di lobby server attivi, che rispondono con 
il proprio nome identificativo.
Durante la fase di registrazione i peer si connettono al 
server fornendo il proprio username e la porta su cui sono in ascolto. Il 
server risponde con nomi utenti, porte e IP degli altri nodi, in modo da 
consentire l'avvio della comunicazione di tipo p2p. In questa fase non vengono 
gestiti eventuali crash dei peer, in quanto tale gestione avverrà nelle fasi 
successive, nelle quali viene implementata la comunicazione distribuita vera e 
propria.
\subsubsection{Scelta del turno}
Durante la scelta del turno ciascun nodo scorre la lista dei propri 
avversari e contatta ognuno di essi per ottenere un numero generato in modo 
pseudocausale. In questo modo i peer sono in possesso di valori associati a 
ciascun avversario, che vengono utilizzati per determinare l'ordine di gioco in 
modo fair. Nel caso in cui due o più valori siano uguali, la precedenza fra di essi 
viene determinata a partire dall'ordinamento alfabetico degli username. In 
seguito alla scelta del turno ciascun nodo è in possesso di una sequenza 
ordinata di giocatori, che viene usata in seguito per scegliere chi debba 
cominciare la partita.
\subsubsection{Gioco}
Può quindi iniziare la fase di gioco vero e proprio. Tutti i peer si 
collegano quindi al giocatore che possiede il turno (turn owner). %FIXME
Quest'ultimo seleziona dunque la propria mossa, la comunica al nodo bersaglio, 
allo scopo di conoscere lo stato della cella selezionata, infine comunica la 
mossa ed il risultato a tutti gli avversari. Per gestire eventuali crash 
da parte di utenti diversi dal turn owner, viene restituita ai nodi anche una 
lista di player crashati, che viene utilizzata in seguito per modificare in 
modo appropriato le strutture dati relative ai giocatori. Oltre al caso di crash 
veri e propri, viene anche gestito il caso in cui la comunicazione di rete a 
partire da qualche giocatore sia troppo latente, per cui viene impostato un 
timer oltre il quale non vengono accettate ulteriori connessioni.
Nel caso in cui sia invece il turn owner a subire un crash, gli altri giocatori 
vengono allertati immediatamente e il gioco può procedere con il nodo 
successivo.
Un'ultima possibilità di errore emerge nel caso in cui il turn owner crashi 
mentre sta restituendo la mossa effettuata ai peer. In tal caso, ci 
si potrebbe trovare in una situazione di inconsistenza dello stato globale del 
sistema. Alcuni nodi conoscerebbero la mossa che il precedente turn owner ha 
effettuato, altri no. È pertanto necessario gestire tali situazioni per 
ripristinare la coerenza globale: al turno successivo sarà il nuovo turn owner 
a confrontare i valori interi contenuti in ciascuna mossa, al fine di 
stabilire quale sia la più recente in ordine cronologico.
Dopo aver modificato il proprio stato se necessario, il turn owner comunica agli 
avversari non aggiornati la mossa più 
recente, prima di iniziare il turno vero e proprio. In questo modo è garantita 
la coerenza dello stato condiviso prima che possa avvenire qualsiasi tipo di 
mossa.


\section{Aspetti Implementativi}
Per l'implementazione del progetto si è scelto di utilizzare il linguaggio 
di programmazione basato sugli oggetti Java, che tramite il framework RMI 
fornisce API specificamente progettate per consentire la realizzazione di 
applicazioni distribuite in modo efficace. Per quanto riguarda l'interfaccia 
grafica, si è utilizzata la libreria di Java Swing.
\\
Il software è stato implementato secondo il pattern architetturale \textbf{MVC},
anche noto come \textbf{Model-View-Controller}, il quale permette di creare una
architettura cosiddetta \textit{multi-tier}, ovvero nella quale le tre
componenti fondamentali di un programma sono suddivise in base ai compiti che
svolgono:
\begin{itemize}
	\item il \textit{modello}: che fornisce i metodi per accedere ai dati utili
	dell'applicazione;
	\item la \textit{vista}: la quale visualizza i dati del modello e si 
occupa
	dell'interazione con l'utente;
	\item il \textit{controller}: che riceve i comandi dall'utente e li attua
	modificando lo stato degli altri due componenti.
\end{itemize}
Nel nostro specifico caso, il modello si compone di tutte quelle classi atte a
mantenere i dati dell'applicazione. Abbiamo dunque l'oggetto che rappresenta il
campo di gioco dell'utente e degli avversari, l'oggetto che rappresenta i
giocatori impegnati nella partita, quello che astrae i dati degli host che si
registrano alla lobby con l'intenzione di partecipare al gioco,
eccetera\dots\newline
La vista è invece formata da tutta una serie di classi atte a permettere la
corretta visualizzazione dei dati, a partire dal frame per la registrazione fino
ad arrivare al frame di gioco.\newline
Vista la natura distribuita del programma, in questo specifico caso il
controller può essere pensato come a tutti quei meccanismi di interazione che
partono dall'utente ed hanno come obiettivo quello di modificare lo
\textit{stato globale} del gioco. Dunque, non soltanto le componenti software
che permettono di interagire direttamente con i dati mantenuti negli oggetti
del mio modello (locale), ma anche tutte le classi che permettono la
comunicazione fra gli host impegnati nella partita.\newline%\subsection{Modello}
Data la natura relativamente semplificata delle regole del gioco, la struttura 
del modello è lineare e ridotta. Per quanto riguarda l'implementazione del 
gioco vera e propria le principali classi coinvolte sono infatti le seguenti:
\begin{itemize}
 \item AField
 \item APlayer
 \item Move
\end{itemize}
Le prime due classi sono astratte, in quanto vengono differenziati il campo di 
gioco proprio al client locale (chiamato MyField) e la propria istanza di 
player(Me). Per quanto riguarda AField, esso rappresenta il campo di gioco e 
contiene pertanto una matrice di interi che memorizza le informazioni 
relative allo stato delle singole celle del campo di battaglia.
La classe astratta APlayer ha due implementazioni: Opponent e Me. Gli 
oggetti di tipo Opponent contengono le informazioni relative agli avversari: ip 
e porta, username e stato dei propri campi di battaglia. La classe Me 
identifica il giocatore locale, memorizza pertanto solo il nome utente e il 
campo del giocatore locale. Le due classi implementano poi alcuni metodi utili 
a conoscere lo stato del giocatore, per sapere ad esempio se egli abbia perso.
L'ultima classe relativa al modello di gioco è Move, che contiene le 
informazioni relative alla mossa effettuata, compreso il risultato di 
quest'ultima. Tale classe è serializzabile, in quanto viene scambiata fra i 
client per comunicare informazioni relative alla mossa scelta e all'ultima 
mossa osservata. Al fine di consentire il recupero di eventuali mosse che non 
siano state trasmesse a tutti i client, ciascuna mossa contiene un intero che 
viene incrementato per ciascuna nuova mossa effettuata.
\\
Oltre alle classi descritte in precedenza, vi è anche una classe NetPlayer, che 
viene utilizzata durante la fase di registrazione per ottenere informazioni 
riguardanti gli altri client dal Lobby Server.
%\subsection{Modello}
Data la natura relativamente semplificata delle regole del gioco, la struttura 
del modello è lineare e ridotta. Per quanto riguarda l'implementazione del 
gioco vera e propria le principali classi coinvolte sono infatti le seguenti:
\begin{itemize}
 \item AField
 \item APlayer
 \item Move
\end{itemize}
Le prime due classi sono astratte, in quanto vengono differenziati il campo di 
gioco proprio al client locale (chiamato MyField) e la propria istanza di 
player(Me). Per quanto riguarda AField, esso rappresenta il campo di gioco e 
contiene pertanto una matrice di interi che memorizza le informazioni 
relative allo stato delle singole celle del campo di battaglia.
La classe astratta APlayer ha due implementazioni: Opponent e Me. Gli 
oggetti di tipo Opponent contengono le informazioni relative agli avversari: ip 
e porta, username e stato dei propri campi di battaglia. La classe Me 
identifica il giocatore locale, memorizza pertanto solo il nome utente e il 
campo del giocatore locale. Le due classi implementano poi alcuni metodi utili 
a conoscere lo stato del giocatore, per sapere ad esempio se egli abbia perso.
L'ultima classe relativa al modello di gioco è Move, che contiene le 
informazioni relative alla mossa effettuata, compreso il risultato di 
quest'ultima. Tale classe è serializzabile, in quanto viene scambiata fra i 
client per comunicare informazioni relative alla mossa scelta e all'ultima 
mossa osservata. Al fine di consentire il recupero di eventuali mosse che non 
siano state trasmesse a tutti i client, ciascuna mossa contiene un intero che 
viene incrementato per ciascuna nuova mossa effettuata.
\\
Oltre alle classi descritte in precedenza, vi è anche una classe NetPlayer, che 
viene utilizzata durante la fase di registrazione per ottenere informazioni 
riguardanti gli altri client dal Lobby Server.
\subsection{Vista}
La vista\dots

\subsection{Comunicazione}
Dal punto di vista implementativo la comunicazione può essere suddivisa a 
partire dalle funzionalità fondamentali che la costituiscono:
\begin{itemize}
 \item Discovery;
 \item Resistrazione;
 \item Gioco.
\end{itemize}
La discovery è quella funzione che consente ai client di trovare eventuali 
lobby attive nella propria sottorete di classe C. La registrazione al Lobby 
Server consente ai client di registrarsi alla partita e di ottenere le 
informazioni fondamentali per l'indirizzamento degli avversari. Le funzionalità 
relative al gioco sono indubbiamente quelle più corpose: vi sono servizi 
finalizzati a stabilire l'ordine di gioco, connettersi al turn owner e 
conoscere lo stato di una cella del campo di gioco avversario durante il 
proprio turno.


\section{Valutazione e Conclusioni}
I test da noi effettuati hanno evidenziato la capacità del software di gestire 
guasti di tipo crash su uno qualsiasi dei client durante la comunicazione. Se 
la gestione di return parziali ha causato un leggero aumento della complessità 
del codice, tali errori non causano più inconsistenze durature, permettendo ai 
client di accordarsi autonomamente rispetto all'ordine delle mosse effettuate.
\\
L'implementazione della discovery delle lobby da parte dei client consente 
inoltre di fornire all'utente una semplice scelta delle stanze di gioco, senza 
assumere che essi conoscano a priori l'indirizzo IP. Naturalmente 
tale funzionalità è utile solo nel caso in cui i client si connettano ad un IP 
della propria sottorete, in quanto è impensabile fare discovery fra reti 
eterogenee senza prevedere un qualche server di appoggio esterno.
\\
Nonostante la semplicità del gioco scelto per il progetto, il protocollo di 
comunicazione implementato può essere facilmente generalizzabile a qualsiasi 
gioco a turni che proceda con una mossa per giocatore in un ordine prestabilito.
Sarebbe infatti sufficiente modificare le strutture dati relative alle mosse 
effettuate e cambiare leggermente le altre parti del codice per implementare un 
qualsiasi nuovo gioco a turni.
\\
Per aumentare la godibilità del gioco si potrebbe infine imporre al giocatore 
di inserire le pecore sul campo di gioco in celle contigue, in modo analogo a 
quanto avviene nella battaglia navale tradizionale. In questo modo ciascun 
attacco che ha avuto successo causerà una ricerca delle pecore rimanenti nelle celle adiacenti,
migliorando la gradevolezza complessiva del gioco.


\end{document}
